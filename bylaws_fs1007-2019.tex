\documentclass[12pt]{scrartcl}

\usepackage[ngerman]{babel}

\title{Satzung des Fachschaftsrates der Japanologie Köln}
\date{Stand: 10.07.2019}

\renewcommand*{\thesection}{§\arabic{section}}

\begin{document}
\maketitle

\section{Mitglieder des Fachschaftsrates}

\begin{enumerate}
	\item Der Fachschaftsrat der Japanologie Köln vertritt alle eingeschriebenen Studierenden der Studiengänge KuGA-Japan, Japanische Kultur in Geschichte und Gegenwart, Japanische Medien- und Populärkultur und Japanisch Lehramt, die der Universität Köln angehören und besteht aus freiwilligen Mitgliedern dieser Studiengänge.
	\item Die Fachschaftsratsmitglieder werden während der Studierendenvollversammlung in den Fachschaftsrat aufgenommen. Der Aufnahmewunsch ist spätestens während der Studierendenvollversammlung dem Fachschaftsvorstand mitzuteilen. In Einzelfällen kann ein Sonderantrag auf Beitritt gestellt werden, über den während der nächsten Ratssitzung abgestimmt wird.
	\item Mitglieder oder Mitgliedskandidaten des Fachschaftsrates können aus wichtigem Grund ohne Einhaltung einer Frist aus dem Fachschaftsrat ausgeschlossen werden, wenn ein Vertrauensmissbrauch oder Tatsachen vorliegen, auf Grund derer dem Fachschaftsrat und der Studierendenschaft unter Berücksichtigung aller Umstände des Einzelfalls und unter Abwägung aller Interessen eine Fortsetzung der Mitgliedschaft oder ein Eintritt in den Fachschaftsrat nicht zugemutet werden kann. Über Ausschlüsse entscheidet die Fachschaftsratssitzung oder die Studierendenvollversammlung mit einer Zweidrittelmehrheit der Anwesenden
\end{enumerate}

\section{Aufgaben des Fachschaftsrates}
\label{sec:aufgaben}

\begin{enumerate}
	\item Dem Fachschaftsrat obliegt die Förderung der Studierendendschaft bei diversen Studienangelegenheiten. Dabei sind besonders Gruppen zu berücksichtigen, die in ihrem Studium mit Schwierigkeiten konfrontiert sind, beispielsweise mithilfe von Tutorien.
	\item Er wirkt bei der Studienberatung mit und nimmt an Akkreditierungsveranstaltungen der Studienordnung teil.
	\item Der Fachschaftsrat sorgt für die soziale Beratung und Betreuung der Studierendenschaft der Japanologie Köln und ist auch Ansprechpartner für japanische Austauschstudierende.
	\item Es ist Aufgabe des Fachschaftsrates die gesamte Arbeit der Fakultät zu überprüfen; gegebenenfalls ist die Beseitigung von Missständen einzuleiten bzw. herbeizuführen.
	\item Dem Fachschaftsrat obliegt die Teilnahme an der Fachschaftenkonferenz.
	\item Er ist verpflichtet die Grundsätze des Datenschutzes bei seiner Arbeit zu befolgen. Des Weiteren hat er diesbezügliche Mängel innerhalb der Fakultät für Japanologie aufzuzeigen.
	\item Dem Fachschaftsrat obliegt die Förderung des Deutsch-Japanischen Austausches an der Universität zu Köln.
\end{enumerate}

\section{Organe der Fachschaft}

\begin{itemize}
	\item Studierendenvollversammlung der Japanologie
	\item Fachschaftsratssitzung
	\item Fachschaftsratsvorstand
\end{itemize}

\section{Studierendenvollversammlung der Japanologie}
\label{par:assembly}

\begin{enumerate}
	\item Die Studierendenvollversammlung findet in der Regel zu Beginn (in der Woche vor Vorlesungsbeginn) und bei Bedarf auch zum Ende der Vorlesungszeit (vor der Klausurphase) statt.
	\item Die Einberufung einer Studierendenvollversammlung geschieht durch Aushang der Tagesordnung über die Informationskanäle des Fachschaftsrates unter Angabe von Ort und Zeit mindestens drei Tage vorher.
	\item Stimmberechtigt auf der Studierendenvollversammlung sind alle anwesenden Studierenden der Japanologie.
	\item Die Studierendenvollversammlung ist nach ordnungsgemäßer Einberufung beschlussfähig. Es müssen allerdings mindestens drei Studierende der Japanologie anwesend sein. Die Studierendenvollversammlung beschließt mit Mehrheit der abgegebenen Stimmen.
	\item\label{par:assembly:tasks} Folgende Aufgaben sind während der Studierendenvollversammlung durchzuführen:
	\begin{itemize}
		\item Wahlen (Die Studierendenvollversammlung kann keine Kandidaten ablehnen) von:
		\begin{itemize}
			\item Fachschaftsratsmitglieder
			\item 1. Fachschaftsratsvorstand
			\item 2. Fachschaftsratsvorstand
			\item Beauftragte Person für Finanzen
			\item Amt, das die Studierenden nach \ref{sec:aufgaben} informiert
			\item Nach Bedarf weitere Ämter
		\end{itemize}
		\item grobe Semesterplanung
		\item Planung wichtiger Veranstaltungen und Events zum Semestereinstieg / Semesterende
	\end{itemize}
	\item Abhängig von der Mitgliederzahl, den Arbeitsbereichen und dem Ausmaß an Verantwortung des Fachschaftsrates in einem Semester kann alternativ zu dem Modell eines 1. und 2. Fachschaftsratsvorstands zur Leitung des Fachschaftsrates ein Vorstandsteam eingesetzt werden.
	\item Die Tagesordnung der Studierendenvollversammlung wird vom Fachschaftsratsvorstand vorbereitet. Alle Studierenden der Japanologie sind berechtigt, Tagesordnungspunkte zu beantragen.
	\item Der Fachschaftsratsvorstand eröffnet die Vollversammlung und stellt die Diskussionsleitung. Die Diskussionsleitung sorgt für eine beim Thema bleibende Diskussion. Sie kann die vorgetragenen Ansichten zusammenfassen und wesentliche Punkte herausarbeiten. Anträge zur Geschäftsordnung sind:
	\begin{itemize}
		\item Antrag auf Nichtbefassung bzw. keine weitere Befassung
		\item Antrag auf Vertagung
		\item Antrag auf Überweisung an einen Ausschuss
		\item Antrag auf Schluss der Debatte
		\item Antrag auf geheime Abstimmung. Einem Antrag auf geheime Abstimmung ist stattzugeben
	\end{itemize}
	\item Über jede Studierendenvollversammlung wird ein Protokoll geführt. Es muss enthalten:
	\begin{itemize}
		\item Die wesentlichen Berichte
		\item Die Ergebnisse der Abstimmungen
		\item Die anwesenden Mitglieder
	\end{itemize}
	Das Protokoll wird von der Protokollführung in dem Dropbox-Ordner der Japanologie gespeichert und an die Mitglieder weitergeleitet. Auf Antrag kann eine Änderung des Protokolls auf der nächsten Fachschaftsratssitzung beschlossen werden.
\end{enumerate}

\section{Fachschaftsratssitzung}

\begin{enumerate}
	\item Die Fachschaftsratssitzung besteht aus den Fachschaftsratsmitgliedern. Sie ist öffentlich und es besteht Rederecht. Unter besonderen Umständen können Ausnahmen vorgenommen werden.
	\item Stimmberechtigt sind alle teilnehmenden Studierenden der Japanologie.
	\item Die Fachschaftsratssitzung stellt den Informationsfluss von anderen universitären Gremien zur Studierendenschaft sicher.
	\item Die Fachschaftsratssitzungen werden einberufen
	\begin{itemize}
		\item während der Vorlesungszeit mindestens zweimal pro Monat vom Fachschaftsratsvorstand
		\item auf Antrag von zwei Fachschaftsratsmitgliedern
		\item bei Vertagung wird dies mindestens einen Tag vorher angekündigt
	\end{itemize}
	\item Über jede Fachschaftsratssitzung ist ein Protokoll zu führen, welches im Anschluss an die anderen Fachschaftsratsmitglieder weiterzuleiten und an einem zentralen Speicherort hochzuladen ist.
	\item Die Fachschaftsratssitzung ist nur beschlussfähig, wenn mindestens drei Fachschaftsratsmitglieder anwesend sind.
	\item Der Fachschaftsratsvorstand gibt der Fachschaftsratssitzung eine Tagesordnung. Jeder Teilnehmende kann Punkte zur Tagesordnung beitragen.
	\item Die Fachschaftsratssitzung beschließt mit Mehrheit der abgegebenen Stimmen (Enthaltungen sind keine abgegebenen Stimmen). Stimmen weniger als zwei Studierende der Japanologie ab, so ist die Abstimmung ungültig und muss wiederholt werden. 
\end{enumerate}

\section{Fachschaftsratsvorstand}

\begin{enumerate}
	\item Der Fachschaftsratsvorstand besteht aus einem 1. und einem 2. Vorstand, die in der Studierendenvollversammlung gewählt werden.
	\item Aufgaben des Fachschaftsratsvorstandes:
	\begin{itemize}
		\item Der Fachschaftsratsvorstand nimmt Kandidaturen für Ämter im Fachschaftsrat an.
		\item Der Fachschaftsratsvorstand überwacht die Durchführung der Beschlüsse von Fachschaftsratssitzung und Studierendenvollversammlung.
		\item Der Fachschaftsratsvorstand koordiniert die Arbeit der Fachschaftsratssitzung mit der Arbeit der studentischen Mitglieder in den Gremien der Universität.
	\end{itemize}
	\item Der Fachschaftsratsvorstand ist der Studierendenvollversammlung rechenschaftspflichtig.
	\item Bei Gleichstand in Abstimmungen entscheidet die Stimme des Fachschaftsratsvorstandes. Sollten 1. und 2. Vorstand beide anwesend sein, hat die Stimme des 1. Vorstandes einen höheren Wert.
	\item Der Fachschaftsratsvorstand ist für die Kommunikation mit dem Institut verantwortlich.
\end{enumerate}

\section{Ämter im Fachschaftsrat}

\begin{enumerate}
	\item Die Fachschaftsratsmitglieder, die Ämter übernehmen, kandidieren freiwillig und werden in der Studierendenvollversammlung gewählt.
	\item Sie haben der Fachschaftsratssitzung laufend Bericht zu erstatten.
	\item Sie sind an die Weisungen der Studierendenvollversammlung und an die Beschlüsse der Fachschaftsratssitzungen gebunden.
	\item Die Neuwahl der Ämter erfolgt in den Studierendenvollversammlungen. Neugewählte Fachschaftsratsmitglieder, die Ämter übernehmen, werden von den Entlassenen eingewiesen.
	\item Eine Abwahl von Ämtern kann von jedem Mitglied vorgeschlagen werden. Die betreffende Person hat das Recht auf eine Stellungnahme. Eine Enthebung des Amtes erfordert einen Beschluss durch eine Abstimmung mit zwei Drittel Mehrheit. Eine solche Abstimmung muss fünf Tage im Vorfeld bekannt gegeben werden. Jedes Fachschaftsratsmitglied ist stimmberechtigt. Die Wahl verläuft geheim. An der Wahl müssen mindestens fünf Personen teilnehmen, sonst ist die Abstimmung ungültig und muss zur nächsten Fachschaftsratssitzung wiederholt werden.
	\item Es werden mindestens die Ämter aus \ref{par:assembly}.\ref{par:assembly:tasks} gewählt.
	\item Ein Fachschaftsratsmitglied scheidet aus dem Amt aus:
	\begin{enumerate}
		\item durch Rücktritt
		\item durch Misstrauenserklärung der Fachschaftsratssitzung
		\item am Ende der Amtszeit (die Amtszeit beträgt ein Semester)
		\item durch Exmatrikulation
	\end{enumerate}
	Bei vorzeitigem Ausscheiden aus dem Amt muss ein neues Fachschaftsratsmitglied kandidieren und von der nächstfolgenden Fachschaftsratssitzung bestätigt werden.
\end{enumerate}

\section{Fachschaftsratsvertretung nach außen}

\begin{enumerate}
	\item Alle Fachschaftsratsmitglieder können den Fachschaftsrat nach außen vertreten. Der Fachschaftsratsvorstand organisiert die Teilnahme von Fachschaftsratsmitgliedern an Sitzungen von universitären Gremien.
	\item Sie sind der Fachschaftsratssitzung rechenschaftspflichtig.
	\item Jedes Jahr muss in einer Fachschaftsratssitzung eine studentische Vertretung und Stellvertretung für den Vorstand im Ostasiatischen Seminar gewählt werden. Diese nehmen an den Vorstandssitzungen teil. 
\end{enumerate}

\section{Finanzen}

\begin{enumerate}
	\item Die beauftragte Person für Finanzen ist für das Rechnungswesen des Fachschaftsrates verantwortlich.
	\item Für Events existiert eine Bargeldkasse. Die Schlüssel dafür verwalten der Fachschaftsratsvorstand und der Finanzbeauftragte.
	\item Die beauftragte Person für Finanzen ist für das Absetzen der Kosten beim ASTA und SpRat verantwortlich.
	\item Der Fachschaftsratsvorstand ist über den Betrag der Bargeldkasse auskunftsberechtigt.
	\item Der Fachschaftsrat darf keinen Gewinn erwirtschaften. Jegliches Einkommen muss wieder zu Gunsten der Studierenden ausgegeben werden. 
\end{enumerate}

\section{Änderung der Satzung des Fachschaftsrates}

\begin{enumerate}
	\item Die Satzung kann auf Antrag eines Fachschaftsratsmitgliedes geändert werden. Die Änderung wird in der Fachschaftsratssitzung besprochen. Dies muss fünf Tage vorher bekannt gegeben werden. Es wird über den Änderungsantrag abgestimmt und es muss eine Beschlussfähigkeit von einer Zweidrittelmehrheit mit mindestens drei teilnehmenden Fachschaftsratsmitgliedern festgestellt werden. Wird Beschlussunfähigkeit festgestellt, so wird die Änderung auf die nächste Sitzung vertagt.
\end{enumerate}

\section{Auflösung des Fachschaftsrates}

\begin{enumerate}
	\item Der Fachschaftsrat der Japanologie Köln ist gemäß der Satzung der Philosophischen Fakultät integraler Bestandteil der Philosophischen Fakultät und kann sich deshalb nicht selbst auflösen. 
\end{enumerate}

\section{Sonstige Bezeichnungen und Begriffe}

\begin{enumerate}
	\item Im alltäglichen Gebrauch sowie auf öffentlichen Kanälen, wie zum Beispiel der Homepage des Fachschaftsrates und den sozialen Netzwerken, wird für den Begriff Fachschaftsrat und allen Bezeichnungen, die sich aus diesem Wort ergeben, das Wort Fachschaft verwendet. 
\end{enumerate}

\end{document}